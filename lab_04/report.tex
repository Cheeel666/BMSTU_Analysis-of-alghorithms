\documentclass[12pt]{report}
\usepackage[utf8]{inputenc}
\usepackage[russian]{babel}
%\usepackage[14pt]{extsizes}
\usepackage{listings}
\usepackage{graphicx}
\usepackage{amsmath,amsfonts,amssymb,amsthm,mathtools} 
\usepackage{pgfplots}
\usepackage{filecontents}
\usepackage{indentfirst}
\usepackage{eucal}
\usepackage{amsmath}
\usepackage{enumitem}
\frenchspacing

\usepackage{indentfirst} % Красная строка


%\usetikzlibrary{datavisualization}
%\usetikzlibrary{datavisualization.formats.functions}

\usepackage{amsmath}




% Для листинга кода:
\lstset{ %
language=haskell,                 % выбор языка для подсветки (здесь это С)
basicstyle=\small\sffamily, % размер и начертание шрифта для подсветки кода
numbers=left,               % где поставить нумерацию строк (слева\справа)
numberstyle=\tiny,           % размер шрифта для номеров строк
stepnumber=1,                   % размер шага между двумя номерами строк
numbersep=5pt,                % как далеко отстоят номера строк от подсвечиваемого кода
showspaces=false,            % показывать или нет пробелы специальными отступами
showstringspaces=false,      % показывать или нет пробелы в строках
showtabs=false,             % показывать или нет табуляцию в строках
frame=single,              % рисовать рамку вокруг кода
tabsize=2,                 % размер табуляции по умолчанию равен 2 пробелам
captionpos=t,              % позиция заголовка вверху [t] или внизу [b] 
breaklines=true,           % автоматически переносить строки (да\нет)
breakatwhitespace=false, % переносить строки только если есть пробел
escapeinside={\#*}{*)}   % если нужно добавить комментарии в коде
}

\usepackage[left=2cm,right=2cm, top=2cm,bottom=2cm,bindingoffset=0cm]{geometry}
% Для измененных титулов глав:
\usepackage{titlesec, blindtext, color} % подключаем нужные пакеты
\definecolor{gray75}{gray}{0.75} % определяем цвет
\newcommand{\hsp}{\hspace{20pt}} % длина линии в 20pt
% titleformat определяет стиль
\titleformat{\chapter}[hang]{\Huge\bfseries}{\thechapter\hsp\textcolor{gray75}{|}\hsp}{0pt}{\Huge\bfseries}


% plot
\usepackage{pgfplots}
\usepackage{filecontents}
\usetikzlibrary{datavisualization}
\usetikzlibrary{datavisualization.formats.functions}

\begin{document}
%\def\chaptername{} % убирает "Глава"
\thispagestyle{empty}

\begin{titlepage}
	\centering
	{\scshape\LARGE МГТУ им. Баумана \par}
	\vspace{3cm}
	{\scshape\Large Лабораторная работа №5\par}
	\vspace{0.5cm}	
	{\scshape\Large По курсу: "Анализ алгоритмов"\par}
	\vspace{1.5cm}
	{\huge\bfseries Параллельное умножение матриц \par}
	\vspace{2cm}
	\Large Работу выполнил: Ковалев Дмитрий ИУ7-53Б\par
	\vspace{0.5cm}
	\LargeПреподаватели:  Волкова Л.Л., Строганов Ю.В.\par

	\vfill
	\large \textit {Москва, 2020} \par
\end{titlepage}

\setcounter{page}{2}
\tableofcontents

\newpage
\chapter*{Введение}
\addcontentsline{toc}{chapter}{Введение}

Многопоточность — способность центрального процессора (CPU) или одного ядра в многоядерном процессоре одновременно выполнять несколько процессов или потоков, соответствующим образом поддерживаемых операционной системой.

Этот подход отличается от многопроцессорности, так как многопоточность процессов и потоков совместно использует ресурсы одного или нескольких ядер: вычислительных блоков, кэш-памяти ЦПУ или буфера перевода с преобразованием (TLB).


В тех случаях, когда многопроцессорные системы включают в себя несколько полных блоков обработки, многопоточность направлена на максимизацию использования ресурсов одного ядра, используя параллелизм на уровне потоков, а также на уровне инструкций.

Поскольку эти два метода являются взаимодополняющими, их иногда объединяют в системах с несколькими многопоточными ЦП и в ЦП с несколькими многопоточными ядрами.


Многопоточная парадигма стала более популярной с конца 1990-х годов, поскольку усилия по дальнейшему использованию параллелизма на уровне инструкций застопорились.

Смысл многопоточности — квазимногозадачность на уровне одного исполняемого процесса.

Значит, все потоки процесса помимо общего адресного пространства имеют и общие дескрипторы файлов. Выполняющийся процесс имеет как минимум один (главный) поток.


Многопоточность (как доктрину программирования) не следует путать ни с многозадачностью, ни с многопроцессорностью, несмотря на то, что операционные системы, реализующие многозадачность, как правило, реализуют и многопоточность.


Достоинства:

\begin{itemize}

	\item облегчение программы посредством использования общего адресного пространства;

	\item меньшие затраты на создание потока в сравнении с процессами;

	\item повышение производительности процесса за счёт распараллеливания процессорных вычислений;

	\item если поток часто теряет кэш, другие потоки могут продолжать использовать неиспользованные вычислительные ресурсы.

\end{itemize}


Недостатки:

\begin{itemize}

	\item несколько потоков могут вмешиваться друг в друга при совместном использовании аппаратных ресурсов \cite{Nemirovsky};

	\item с программной точки зрения аппаратная поддержка многопоточности более трудоемка для программного обеспечения \cite{Olukotun};

	\item проблема планирования потоков;

	\item специфика использования. Вручную настроенные программы на ассемблере, использующие расширения MMX или AltiVec и выполняющие предварительные выборки данных, не страдают от потерь кэша или неиспользуемых вычислительных ресурсов. Таким образом, такие программы не выигрывают от аппаратной многопоточности и действительно могут видеть ухудшенную производительность из-за конкуренции за общие ресурсы.

\end{itemize}

Однако несмотря на количество недостатков, перечисленных выше, многопоточная парадигма имеет большой потенциал на сегодняшний день и при должном написании кода позволяет значительно ускорить однопоточные алгоритмы.

\section*{Цель лабораторной работы}
Целью данной лабораторной работы является изучение и реализация параллельных вычислений.

\section*{Задачи лабораторной работы}
\indent В рамках выполнения работы необходимо решить следующие задачи:

\begin{itemize}

	\item изучить понятие параллельных вычислений;

	\item реализовать последовательный и параллельный алгоритм перемножения матриц;

	\item сравнить временные характеристики реализованных алгоритмов экспериментально.
\end{itemize}

\chapter{Аналитическая часть}

\section{Описание задачи}


Пусть даны две прямоугольные матрицы
\begin{equation}
A_{lm} = \begin{pmatrix}
a_{11} & a_{12} & \ldots & a_{1m}\\
a_{21} & a_{22} & \ldots & a_{2m}\\
\vdots & \vdots & \ddots & \vdots\\
a_{l1} & a_{l2} & \ldots & a_{lm}
\end{pmatrix},
\quad
B_{mn} = \begin{pmatrix}
b_{11} & b_{12} & \ldots & b_{1n}\\
b_{21} & b_{22} & \ldots & b_{2n}\\
\vdots & \vdots & \ddots & \vdots\\
b_{m1} & b_{m2} & \ldots & b_{mn}
\end{pmatrix},
\end{equation}

тогда матрица $C$
\begin{equation}
C_{ln} = \begin{pmatrix}
c_{11} & c_{12} & \ldots & c_{1n}\\
c_{21} & c_{22} & \ldots & c_{2n}\\
\vdots & \vdots & \ddots & \vdots\\
c_{l1} & c_{l2} & \ldots & c_{ln}
\end{pmatrix},
\end{equation}

где
\begin{equation}
\label{eq:M}
c_{ij} =
\sum_{r=1}^{m} a_{ir}b_{rj} \quad (i=\overline{1,l}; j=\overline{1,n})
\end{equation}

будет называться произведением матриц $A$ и $B$.


В данной лабораторной работе стоит задача распараллеливания алгоритма Винограда. Так как каждый элемент матрицы $C$ вычисляется независимо от других и матрицы $A$ и $B$ не изменяются, то для параллельного вычисления произведения, достаточно просто равным образом распределить элементы матрицы $C$ между потоками.

\section{Вывод}
	Обычный алгоритм перемножения матриц независимо вычисляет элементы матрицы-результата, что дает большое количество возможностей для реализации параллельного варианта алгоритма.
\clearpage

\chapter{Конструкторская часть}

На рисунке 2.1 представлена схема обычного алгоритма перемножения матриц (без распараллеливания). На рисунках 2.2 и представлены схема распараллеливания алгоритма умножения матриц.

\section{Схемы алгоритмов}


\begin{figure}[h]
	\centering
	\includegraphics[scale=0.85]{base.jpg}
	\caption{Схема стандартного алгоритма умножения матриц.}
	\label{fig:mpr}
\end{figure}

\begin{figure}[h]
	\centering
	\includegraphics[scale=0.85]{parallel1.jpg}
	\caption{Схема распараллеленного алгоритма умножения матриц}
	\label{fig:mpr}
\end{figure}

\begin{figure}[h]
	\centering
	\includegraphics[scale=1]{start_threading.jpg}
	\caption{Функция создания потоков и запуска параллельных реализация умножения матриц}
	\label{fig:mpr}
\end{figure}

На рисунке 2.4 представленна схема с параллельным выполнением первого цикла (по строкам матрицы)

\begin{figure}[h]
	\centering
	\includegraphics[scale=0.7]{parallel_scheme_01.jpg}
	\caption{Схема с параллельным выполнением первого цикла}
	\label{fig:mpr}
\end{figure}

\section{Вывод}
На основе теоретических данных, полученных из аналитического раздела, была построена схема стандартного алгоритма умножения матриц, а так же после разделения алгоритма на этапы были предложены 2 схемы параллельного выполнения данных этапов.

\chapter{Технологическая часть}

В данном разделе приведены средства реализации и листинги кода.

\section{Требование к ПО}

К программе предъявляется ряд требований:

\begin{itemize}
	\item на вход подаются размеры 2 матриц, а также их элементы;
	\item на выходе — матрица, которая является результатом умножения входных матриц.
\end{itemize}

\section{Средства реализации}
Для реализации ПО я выбрал язык программирования C++. Данный выбор обусловлен высокой скоростью работы языка, а так же наличия инструментов для создания и эффективной работы с потоками.

\section{Реализация алгоритмов}

В листингах 3.1 - 3.3 приведена реализация расмотренных ранее алгоритмов перемножения матриц.
В листинге 3.4 приведена реализация функции создания и распределения потоков.

\begin{lstlisting}[label=some-code,caption=Функция умножения матриц обычным способом, language=C]
void multiMatrix(double **a, double **b, double **c, int row, int col) {
    for (int i = 0; i < row; i++) {
        for (int j = 0; j < col; j++) {
            c[i][j] = 0;
            for (int k = 0; k < col; k++)
                c[i][j] += a[i][k] * b[k][j];
        }
    }
}
\end{lstlisting}

\begin{lstlisting}[label=some-code,caption=Функция умножения матриц параллельно. language=C]
void iteration(double **a, double **b, double **c, int row, int col, int th_number, int th_amount) {
    for (int i = th_number; i < row; i += th_amount) {
        for (int j = 0; j < col; j++) {
            c[i][j] = 0;
            for (int k = 0; k < col; k++)
                c[i][j] += a[i][k] * b[k][j];
        }
    }
}
\end{lstlisting}

\begin{lstlisting}[label=some-code,caption=Функция создания потоков,language=C]
void multiMatrixParallel(double **a, double **b, double **c, int row, int col, int n) {
    std::thread workers[n];
    for (int i = 0; i < n; i++) {
        workers[i] = std::thread(iteration, a, b, c, row, col, i, n);
    }
    for (int i = 0; i < n; i++) {
        workers[i].join();
    }
}
\end{lstlisting}

\section{Тестовые данные}

В таблице~\ref{tabular:test_rec} приведены тесты для функций, реализующих параллельное и обычное умножение матриц. Все тесты пройдены успешно.

\begin{table}[h!]
	\begin{center}
		\begin{tabular}{c@{\hspace{7mm}}c@{\hspace{7mm}}c@{\hspace{7mm}}c@{\hspace{7mm}}c@{\hspace{7mm}}c@{\hspace{7mm}}}
			\hline
			Первая матрица & Вторая матрица & Ожидаемый результат \\ \hline
			\vspace{4mm}
			$\begin{pmatrix}
			1 & 2 & 3\\
			1 & 2 & 3\\
			1 & 2 & 3
			\end{pmatrix}$ &
			$\begin{pmatrix}
			1 & 2 & 3\\
			1 & 2 & 3\\
			1 & 2 & 3
			\end{pmatrix}$ &
			$\begin{pmatrix}
			6 & 12 & 18\\
			6 & 12 & 18\\
			6 & 12 & 18
			\end{pmatrix}$ \\
			\vspace{2mm}
			\vspace{2mm}
			$\begin{pmatrix}
			1 & 2 & 2\\
			1 & 2 & 2
			\end{pmatrix}$ &
			$\begin{pmatrix}
			1 & 2\\
			1 & 2\\
			1 & 2
			\end{pmatrix}$ &
			$\begin{pmatrix}
			5 & 10\\
			5 & 10
			\end{pmatrix}$ \\
			\vspace{2mm}
			\vspace{2mm}
			$\begin{pmatrix}
			2
			\end{pmatrix}$ &
			$\begin{pmatrix}
			2
			\end{pmatrix}$ &
			$\begin{pmatrix}
			4
			\end{pmatrix}$ \\
			\vspace{2mm}
			\vspace{2mm}
			$\begin{pmatrix}
			1 & -2 & 3\\
			1 & 2 & 3\\
			1 & 2 & 3
			\end{pmatrix}$ &
			$\begin{pmatrix}
			-1 & 2 & 3\\
			1 & 2 & 3\\
			1 & 2 & 3
			\end{pmatrix}$ &
			$\begin{pmatrix}
			0 & 4 & 6\\
			4 & 12 & 18\\
			4 & 12 & 18
			\end{pmatrix}$\\
			\vspace{2mm}
			\vspace{2mm}
		\end{tabular}
	\end{center}
	\caption{\label{tabular:test_rec} Тестирование функций}
\end{table}

\section{Вывод}

В данном разделе были разработаны исходные коды алгоритмов: обычный способ умножения матриц и два способа параллельного перемножения матриц.

\chapter{Исследовательская часть}

\section{Технические характеристики}

Ниже приведены технические характеристики устройства, на котором было проведено тестирование ПО:

\begin{itemize}
	\item Операционная система: Debian \cite{debian} Linux \cite{linux} 11 <<bullseye>> 64-bit.
	\item Оперативная память: 12 GB.
	\item Процессор: Intel(R) Core(TM) i5-3550 CPU @ 3.30GHz
\cite{i5}.

\end{itemize}

\section{Время выполнения алгоритмов}
В таблице 4.1 приведено сравнение двух реализаций параллельного умножения матриц с разным количеством потоков при перемножении матриц размером 512. В таблице 4.2 приведено сравнение однопоточной реализации и многопоточных (с предварительным транспонированием и без) (на четырёх потоках).

\begin{table} [h!]
	\caption{Таблица времени выполнения параллельных алгоритмов, при размерах перемножаемых матриц 512 (в тиках)}
	\begin{center}
		\begin{tabular}{|c c c c c c|} 
			\hline
			Размер матрицы & 1 поток & 2 потока & 4 потока & 8 потоков & 16 потоков\\  
			\hline
			128 & 21 752 & 16 242 & 12 477  & 16 472 & 20 233\\
			\hline
			256 & 172 266 & 127 088 & 85 094 & 172 544 & 186 473 \\
			\hline
			512 & 1 552 812 & 1 245 364 & 943 807 & 1 713 305 & 1 720 823 \\
			\hline
			1024 & 26 584 110 & 18 450 302 & 12 547 271 & 19 919 997 & 21 114 917 \\
			\hline
		\end{tabular}
	\end{center}
\end{table}

\section{Вывод}

Наилучшее время параллельные алгоритмы показали на 4 потоках, что соответствует количеству логических ядер компьютера, на котором проводилось тестирование. На матрицах размером 512 на 512, параллельные алгоритмы улучшают время обычной (однопоточной) реализации перемножения матриц примерно в 1.7 раза. При количестве потоков, большее чем 4, параллельная реализация замедляет выполнение (в сравнении с 4 потоками). Кроме этого, если рассматривать транспонирование матриц, то можно получить улучшение в среднем на 15\% быстрее обычного параллельного умножения. Это можно объяснить кэшированием элементов.

\chapter*{Заключение}
\addcontentsline{toc}{chapter}{Заключение}

В рамках данной лабораторной работы была достигнута её цель: изучены паралелльные вычисления. Также выполнены следующие задачи:

\begin{itemize}
	\item было изучено понятие параллельных вычислений;
	\item были реализованы обычный и  параллельный алгоритм перемножения матриц;
	\item было произведено сравнение временных характеристик реализованных алгоритмов экспериментально.
\end{itemize}

Параллельные реализации алгоритмов выигрывают по скорости у обычной (однопоточной) реализации перемножения двух матриц. Наиболее эффективны данные алгоритмы при количестве потоков, совпадающем с количеством логических ядер компьютера. Так же стоит отметить, что при дополнительном транспонировании второй матрицы, можно уменьшить время выполнения параллельных алгоритмов в среднем на 15\%, что объясняется кэшированием элементов.

\addcontentsline{toc}{chapter}{Литература}

\bibliographystyle{utf8gost705u}  % стилевой файл для оформления по ГОСТу

\bibliography{51-biblio}          % имя библиографической базы (bib-файла)


\end{document}
